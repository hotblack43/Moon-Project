\documentclass[12pt,a4paper]{article}

%% Language and font encodings
\usepackage[english]{babel}
\usepackage[T1]{fontenc}
\usepackage{natbib}
\usepackage[utf8]{inputenc}
\usepackage{amssymb}
\usepackage{lineno}
\usepackage{rotating}
\usepackage{datetime2}

\title{Moon Project}
\author{}
\date{February 2019}

\begin{document}

\maketitle

\section{Introduction}

Earthshine is the name for the short-wave light that Earth reflects into space that falls on the Moon on the lunar dark side, as seen from Earth. The photometric study of earthshine intensity is a way to determine terrestrial mean albedo. With long time series of earthshine observations, an impact on climate-change science becomes available because the data are completely independent of the observations performed from orbiting satellites,  and can be obtained more economically than from space.

The analysis of earthshine images is presently done by extracting photometric information from lunar images obtained telescopically. The terrestrial albedo can be determined by basically using the ratio between the dark-side and bright-side illuminations, and a reflectance model that includes sunlight reflected from the Moon as well as light reflected once from the Earth. Reduction methods are applied to remove the bright-side scattered light that pollutes the dark side of the lunar disk. 

We want to extract the terrestrial albedo more directly from images by essentially using machine-learning methods based on supervised training on large sets of the reflectance-model based images, tuned to the geometry of the observations (i.e. image size and orientation, and orbital geometries including lunar libration) with light-scattering parameterized and realistic levels of uniform air-glow included. Training the ML system on training images densely sampling the relevant parameter spaces could yield very direct methods of subsequent data-reduction.

In this paper we report...  




A test citation~\citep{2004sci...304.1299p}.



\bibliographystyle{ametsoc}

\bibliography{dtucompute}

\end{document}
